\documentclass[11pt]{article}
\usepackage{amsmath}
\usepackage{hyperref}
\usepackage{colortbl}
\usepackage[normalem]{ulem} % For strikethrough text
\usepackage{xcolor}         % For text colors


\title{An Introduction to Stirling’s Number of the \\Second Kind}
\author{ChatGPT \& S. H.}
\date{September 21, 2024}

\begin{document}

\maketitle

\section{Introduction}
In combinatorics, \textit{Stirling’s number of the second kind} $S(n, k)$ is the number of ways to \underline{partition a set of $n$ elements into $k$ non-empty subsets} \cite{wiki}. These numbers arise in various areas of mathematics and have applications in \textbf{set theory, number theory, and even computer science}\footnote{https://en.wikipedia.org/wiki/Stirling_numbers_of_the_second_kind}.

Stirling’s numbers of the second kind can be defined recursively and have many interesting properties, which we will explore in this document.

\section{Properties of Stirling Numbers}



\subsection{Definition}
The \texttt{Stirling number of the second kind}, denoted by $S(n, k)$, is defined as the number of ways to divide a set of $n$ elements into $k$ non-empty subsets. It can be written recursively as $S(n, k) = k \cdot S(n-1, k) + S(n-1, k-1)$, for $n > 0$, with the boundary conditions $S(0,0) = 1$, $S(n, 0) = 0$ for $n > 0$, and $S(n, k) = 0$ for $k > n$.

\subsection{Combinatorial Interpretation}
Stirling numbers of the second kind have a natural combinatorial interpretation. They count the ways to partition a set of $n$ elements into $k$ non-empty subsets. For example, consider the set $\{1, 2, 3\}$. The number of ways to partition this set into two subsets is given by $S(3, 2) = 3$. These partitions are:

\begin{itemize}
    \item $\{1\}, \{2, 3\}$ 
    \item ${\{2\}, \{1, 3\}$
    \item $\{3\}, \{1, 2\}$
\end{itemize}

\subsection{Closed Form}
Stirling numbers of the second kind can be described by the following equation:

\textbf{INSERT THE EQUATION SHOWN IN THE PROJECTOR}

\subsection{First Few Examples}
Table 1 shows the values of the Stirling numbers of the second kind, $S(n, k)$, for small values of $n$ and $k$:

\begin{table}[h]
    \centering
    \begin{tabular}{|c|c|c|c|c|c|}
        \hline
        $n \backslash k$ & 1 & 2 & 3 & 4 & 5 \\ \hline
        1 & 1 &  &  &  &  \\ 
        %\hline
        2 & 1 & 1 &  &  &  \\ 
        %\hline
        3 & 1 & 3 & 1 &  &  \\ 
        %\hline
        4 & 1 & 7 & 6 & 1 &  \\ 
        %\hline
        5 & 1 & 15 & 25 & 10 & 1 \\ 
        \hline
    \end{tabular}
    \caption{Stirling Numbers of the Second Kind for $n \leq 5$.}
    \label{tab:stirling_numbers}
\end{table}


\section{Conclusion}

\textcolor{red}{\sout{"Don’t forget to practice more problems involving Stirling numbers to fully understand their applications!"}}


\begin{thebibliography}{9}
\bibitem{wiki} 
R. L. Graham, D. E. Knuth, and O. Patashnik, 
\textit{Concrete Mathematics: A Foundation for Computer Science}. USA: Addison-Wesley Longman Publishing Co., Inc., 2nd ed., 1994.
\end{thebibliography}

\end{document}
