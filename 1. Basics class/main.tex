\documentclass{article}
\usepackage{graphicx} % Required for inserting images
\usepackage{multicol}   
% for adding multi coln or multi rows in document 
\usepackage{multirow}
\usepackage{float}

\usepackage{amsmath}   % to use \begin{equation}  or \begin{align} or \begin{cases}
\title{Demo}
\author{Tanjeem Azwad Zaman}
\date{\today}

\begin{document}

\maketitle
\pagebreak
\tableofcontents
\pagebreak
\listoftables
\pagebreak
% \input{}
\section{Introduction}
    \subsection{Section1.1}
        This is a subsection. \textbf{Bold text}. \textbf{Quickbold}. \textit{italics}. \ref{tab:table1}\emph{this will be emphasized}. \underline{undelined}
    \subsection{fonts}
        \Large{Large Text}. {\huge Huge text} \small{small text} {\small little text }
\section{Section2}
%% Unlisted section ! Use stars for this
\section*{Unlisted Section}
% "itemize" for unordered.. and "enumerate" for ordered . and "tabular " for table adds and use "table" to use different styling for table like caption or centering a table.. and "equation" for math stuffs.. or USE "$$ $$" for math env .
\section{Lists}
    \subsection{Unordered Lists}
        \begin{itemize}
            \item first item
            \item[----] second %% CHECK that we repplace bullet emoji here ! 
            \item [\textbf{Description}] This is the description

            \begin{itemize}
                \item nested item 1
            \end{itemize}
        \end{itemize}
    \subsection{Ordered List}
        \begin{enumerate}
            \item item 1
            \begin{enumerate}
                \item nested item 1
            \end{enumerate}
            \begin{itemize}
                \item unordered
            \end{itemize}
        \end{enumerate}
        
\section{Tables}
    \subsection{Basic Tabular}
        % \begin{center}
        \begin{tabular}{ | l | c | r | }   %% we can use { l | c | r}  ..it means left center and right aligned of those colns
        \hline
         cell1 & cell2 & cell3 \\ 
         \hline
         bigcellcell4 & bigcellcell5 & bigcellcell6 \\  
         \cline{1-2}    %% MORE customize of horizontal lines
         cell7 & cell8 & cell9    \\
         \hline
        \end{tabular}
        % \end{center}
    \subsection{Multi Column}
        % \begin{center}
        \begin{tabular}{ | l | c | r | }
        \hline
         cell1 & cell2 & cell3 \\ 
         \hline
         \multicolumn{2}{|c|}{Merged cols} & bigcellcell6 \\   % check out the syntax of multicolumn. Here 2 means koeta coln merge.. |c| format for the new merged one .. instead of |c| , u can use star (*) . this will take the default auto format
         \hline
         1 & 2 & 3 \\
         \cline{1-2}
         cell7 & cell8 & cell9    \\
         \hline
        \end{tabular}
        % \end{center}
    \subsection{Multi Row}
        \begin{tabular}{|c|c|c|}
            \hline
            c11 & c12 & c13 \\
             \hline
            \multirow{2}{*}{c23} & c22 & c23 \\
            % \hline
            \cline{2-3}
             & c32 & c33 \\
            \hline
        \end{tabular}
    \subsection{custom table}
        % \begin{tabular}{|l|ll|}
        % \hline
        % C1                               & \multicolumn{1}{l|}{C2}                 & C3    \\ \hline
        % \multirow{2}{*}{{\underline underline} & \multicolumn{1}{l|}{\textbf{Bold Texr}} & dummy \\ \cline{2-3} 
        %                                  & \multicolumn{2}{r|}{\textit{ita}}               \\ \hline
        % \end{tabular}
    \begin{table}[H]
    %htbp ..h means jekhane thakar ktha sekhanei thakbe... t for top..b for bottom.. p for new page e draw
        \centering
       \begin{tabular}{|c|c|c|}
                \hline
                c11 & c12 & c13 \\
                 \hline
                \multirow{2}{*}{c23} & c22 & c23 \\
                % \hline
                \cline{2-3}
                 & c32 & c33 \\
                \hline
            \end{tabular}
        \caption{Example table}
        \label{tab:table1} % how to use now as reference.. use \ref{}
             
    \end{table}
    \subsection{Multi Row}
     % follow this link for easier.. https://www.tablesgenerator.com/
        \begin{tabular}{|c|c|c|}
            \hline
            c11 & c12 & c13 \\
             \hline
            \multirow{2}{*}{c23} & c22 & c23 \\
            % \hline
            \cline{2-3}
             & c32 & c33 \\
            \hline
        \end{tabular}



%%      Format 1: \begin{equation}  -- pashapashi ashbena alada line e chole jbe.
%%      Format 2: $ $           --  space dhorte parena.. $ a b c$,,  in document 'abc' show krbe
%%      Format 3:   \begine{align} -- multiple lines can be added easily for proof type things..and last line e sudhu numbering ta ashbe . use 'nonnumber' too careful
%%     \mathcal{}  %% for fancy letters of math
\section{Equations}
    \subsection{Basics}
    This is an equation demo. We have the variables $X$ and $Y_{ij}$ and $Z^{5}$. $a \qquad b \; c$
        \begin{equation}
            X = Y_{ij} + Z^{35}
        \end{equation}
        
        \begin{equation*}
            a = \frac{cd}{ef}
        \end{equation*}

        \begin{align}
            x 
            & = a + a + b + c + c + c \nonumber \\   %% use 'nonumber' otherwise proti line e numbering eshe porbe here
            & =  2a + b + 3c
        \end{align}

        $$a = \frac{cd}{ef}$$
 % type-1: use single dollar notation to automatically number the equations. also careful it will take you to new line.
% type-2: use double dollar sign..it will take u to new line but no numbering.
        $$
        \mathcal{N}(x_i, \mu, \sigma) = \frac{1}{2\sqrt{\pi}\sigma} e ^{-\frac{(x-\mu)^{2}}{2\sigma^2} }
        $$
        $$
        \mathcal{}
        $$
        $$\mathcal{N}(x_i; \mu, \sigma) = \frac{1}{2\sqrt{\pi}\sigma^2} e^{-\frac{(x_i-\mu)^2}{2\sigma^2}}$$


        $$|x| = \begin{cases}
                    x & \text{if } x \geq 0 \\
                    -x & \text{if } x < 0
                \end{cases}$$
            %% '&' this symbol is used to align all the parts in same down down that are after '&' 
\end{document}
